
%% bare_conf.tex
%% V1.3
%% 2007/01/11
%% by Michael Shell
%% See:
%% http://www.michaelshell.org/
%% for current contact information.
%%
%% This is a skeleton file demonstrating the use of IEEEtran.cls
%% (requires IEEEtran.cls version 1.7 or later) with an IEEE conference paper.
%%
%% Support sites:
%% http://www.michaelshell.org/tex/ieeetran/
%% http://www.ctan.org/tex-archive/macros/latex/contrib/IEEEtran/
%% and
%% http://www.ieee.org/

%%*************************************************************************
%% Legal Notice:
%% This code is offered as-is without any warranty either expressed or
%% implied; without even the implied warranty of MERCHANTABILITY or
%% FITNESS FOR A PARTICULAR PURPOSE!
%% User assumes all risk.
%% In no event shall IEEE or any contributor to this code be liable for
%% any damages or losses, including, but not limited to, incidental,
%% consequential, or any other damages, resulting from the use or misuse
%% of any information contained here.
%%
%% All comments are the opinions of their respective authors and are not
%% necessarily endorsed by the IEEE.
%%
%% This work is distributed under the LaTeX Project Public License (LPPL)
%% ( http://www.latex-project.org/ ) version 1.3, and may be freely used,
%% distributed and modified. A copy of the LPPL, version 1.3, is included
%% in the base LaTeX documentation of all distributions of LaTeX released
%% 2003/12/01 or later.
%% Retain all contribution notices and credits.
%% ** Modified files should be clearly indicated as such, including  **
%% ** renaming them and changing author support contact information. **
%%
%% File list of work: IEEEtran.cls, IEEEtran_HOWTO.pdf, bare_adv.tex,
%%                    bare_conf.tex, bare_jrnl.tex, bare_jrnl_compsoc.tex
%%*************************************************************************

% *** Authors should verify (and, if needed, correct) their LaTeX system  ***
% *** with the testflow diagnostic prior to trusting their LaTeX platform ***
% *** with production work. IEEE's font choices can trigger bugs that do  ***
% *** not appear when using other class files.                            ***
% The testflow support page is at:
% http://www.michaelshell.org/tex/testflow/



% Note that the a4paper option is mainly intended so that authors in
% countries using A4 can easily print to A4 and see how their papers will
% look in print - the typesetting of the document will not typically be
% affected with changes in paper size (but the bottom and side margins will).
% Use the testflow package mentioned above to verify correct handling of
% both paper sizes by the user's LaTeX system.
%
% Also note that the "draftcls" or "draftclsnofoot", not "draft", option
% should be used if it is desired that the figures are to be displayed in
% draft mode.
%
\documentclass[conference]{IEEEtran}

\usepackage{url, fancyvrb, framed, multirow, tabularx, graphicx, epstopdf, enumerate, array, cite, algorithmic, fixltx2e}

\usepackage[cmex10]{amsmath}
%\usepackage{breqn}

% correct bad hyphenation here
\hyphenation{op-tical net-works semi-conduc-tor}

\begin{document}
%
% paper title
% can use linebreaks \\ within to get better formatting as desired
\title{NDN Repo: An NDN Persistent Storage Model}


% author names and affiliations
% use a multiple column layout for up to three different
% affiliations
%\author{\IEEEauthorblockN{Michael Shell}
%\IEEEauthorblockA{School of Electrical and\\Computer Engineering\\
%Georgia Institute of Technology\\
%Atlanta, Georgia 30332--0250\\
%Email: http://www.michaelshell.org/contact.html}
%\and
%\IEEEauthorblockN{Homer Simpson}
%\IEEEauthorblockA{Twentieth Century Fox\\
%Springfield, USA\\
%Email: homer@thesimpsons.com}
%\and
%\IEEEauthorblockN{James Kirk\\ and Montgomery Scott}
%\IEEEauthorblockA{Starfleet Academy\\
%San Francisco, California 96678-2391\\
%Telephone: (800) 555--1212\\
%Fax: (888) 555--1212}}

% conference papers do not typically use \thanks and this command
% is locked out in conference mode. If really needed, such as for
% the acknowledgment of grants, issue a \IEEEoverridecommandlockouts
% after \documentclass

% for over three affiliations, or if they all won't fit within the width
% of the page, use this alternative format:
%
%\author{\IEEEauthorblockN{Michael Shell\IEEEauthorrefmark{1},
%Homer Simpson\IEEEauthorrefmark{2},
%James Kirk\IEEEauthorrefmark{3},
%Montgomery Scott\IEEEauthorrefmark{3} and
%Eldon Tyrell\IEEEauthorrefmark{4}}
%\IEEEauthorblockA{\IEEEauthorrefmark{1}School of Electrical and Computer Engineering\\
%Georgia Institute of Technology,
%Atlanta, Georgia 30332--0250\\ Email: see http://www.michaelshell.org/contact.html}
%\IEEEauthorblockA{\IEEEauthorrefmark{2}Twentieth Century Fox, Springfield, USA\\
%Email: homer@thesimpsons.com}
%\IEEEauthorblockA{\IEEEauthorrefmark{3}Starfleet Academy, San Francisco, California 96678-2391\\
%Telephone: (800) 555--1212, Fax: (888) 555--1212}
%\IEEEauthorblockA{\IEEEauthorrefmark{4}Tyrell Inc., 123 Replicant Street, Los Angeles, California 90210--4321}}




% use for special paper notices
%\IEEEspecialpapernotice{(Invited Paper)}




% make the title area
\maketitle


\begin{abstract}
%\boldmath

\end{abstract}
% IEEEtran.cls defaults to using nonbold math in the Abstract.
% This preserves the distinction between vectors and scalars. However,
% if the conference you are submitting to favors bold math in the abstract,
% then you can use LaTeX's standard command \boldmath at the very start
% of the abstract to achieve this. Many IEEE journals/conferences frown on
% math in the abstract anyway.

% no keywords




% For peer review papers, you can put extra information on the cover
% page as needed:
% \ifCLASSOPTIONpeerreview
% \begin{center} \bfseries EDICS Category: 3-BBND \end{center}
% \fi
%
% For peerreview papers, this IEEEtran command inserts a page break and
% creates the second title. It will be ignored for other modes.
\IEEEpeerreviewmaketitle

\section{Introduction}

Data distribution is the largest network stream in current Internet. Many web applications like video or image distribution adopts content delivery network (CDN) for quick access of network data. Besides, more and more data are stored in network storage such as Dropbox, Google Drive and Amazon S3. Network and backend storage are commonly separately designed in current TCP/IP network architecture and there is always network packet and storage data presentation transforming process. In this paper, an in-network storage model Named Data Networking repository (\emph{NDN repo}) is proposed based on NDN network. In NDN, network packet is directly identified by the name of the data, not the source or destination identification. Thanks to this mechanism, \emph{NDN repo} directly storages network and application ready data, which realizes the concept of application level framing. \cite{clark1990architectural}

An \emph{NDN repo} is a set of storage application over NDN network managed by single party. It is upon application level without tweaking the NDN protocol. Compared with \emph{Content Store} (\emph{CS}) which provides data packets in-network cache, \emph{NDN repo} provides persistent storage for data objects. The storage of unit is data object and the management scope is based on NDN name prefix. An NDN repo should response interests with managed name prefix with data packets. Besides, NDN repo also offers data object insertion and deletion, fetching data with assigned name prefix and status check of such progresses. An \emph{NDN repo} should conform to the \emph{NDN repo} protocol, which specifies the semantics of NDN packet for repo interaction and basic processes of each functions, such as insertion and deletion. Repo protocol does not limit packet transport control or any security policy.

In this paper, the design of \emph{NDN repo} protocol is demonstrated. The major design goals are: security for remote operation, management and control based on namespace, reliable operation of large data object which can not be framed in a single data packet. Different special cases are detialedly discussed. An initial \emph{NDN repo} implementation \emph{repo-ng} is also demonstrated. It conforms to the protocol and provides interfaces to configure security policy. The boundary of network and storage is broken that data object could be directly used for application and network transportation. Besides, \emph{NDN repo} protocol can be applied for NDN application using data storage service.

The rest of this paper is organized as follows. Section \ref{section-background} introduce the background of \emph{NDN repo}.  Section \ref{section-design} illustrates the design goals of \emph{NDN repo} and how it works. Section \ref{section-implementation} demonstrates the example implementation of \emph{NDN repo} -- \emph{repo-ng}. Section \ref{section-evaluation} evaluates the performance and functionality of \emph{repo-ng}. Section \ref{section-discussion} discusses the evaluation results and tradeoff in \emph{NDN repo} design. Section \ref{section-conclusion} concludes the paper and addresses the future work.

\section{Background} \label{section-background}
\subsection{Named Data Networking}
Named Data Networking (NDN) \cite{zhang2010named} is a data-oriented network architecture which replaces IP with names of data packets as the narrow waist of networking. The essential evolution of NDN is the change of network behaviors from delivering data to a certain destination to fetching data with a given name. \cite{zhang2010named} Because of this change, \emph{interest} and \emph{data object} are imported as network packets for fetching and responding data of given name. \emph{Interest} is the request of network packets containing prefix of names and other constraints. Data packet contains the content of data and the digital signature signed by data producer.

In-network storage means device at network level can not only cache network packet temporarily, but also stores the data packet for local or remote application using directly. NDN makes in-network storage possible because of following reasons: Network packet is identified by source and destination host addresses in IP network, while name of NDN packet is irrelevant with physical endpoints. Any host in NDN network carrying data of given names can response to the \emph{interest}, but hosts besides source and destination cannot retrieve in-network IP packet. Another concern is the privacy of in-network data. Signature in data packet is to resolve authentication and confidentiality of data. \emph{Content Store (CS)} is cache of data packet in NDN router model and it is within network layer.

In \cite{clark1990architectural}, TCP/IP and other layered network architecture are revisited. Data presentation is major cost in network data processing. This presentation process should not be limited to the current network function, but should adjust to certain application wants. Concept of application level framing is proposed to move presentation and transport control from network to application level. This principle is fully adopted in \emph{NDN repo} design.

\subsection{ccnr}
 \emph{Ccnr} is a subset software of CCNx project\footnote{CCNx: http://www.ccnx.org/what-is-ccn/}. CCN and NDN have the same origin and their designs of architecture are similar. \emph{Cnnr} is a repository that preserving CCN network packet. It supports remote data fetching using interests and local data dump. The functions of \emph{ccnr} is limited. User cannot issue data insertion command remotely and command cannot be verified to apply security policy. This repository protocol is not sufficient for application needs.

\section{Design of NDN Repo protocol} \label{section-design}

\subsection{overall design of repo protocol}

The goal of this section is to demonstrate protocol and discuss concerns on potential failure cases.

NDN Repo protocol is specification of NDN network packet and process to operate \emph{NDN repo}. Controls of the network transportation such as flow control, access control and so on are all defined in \emph{repo} application level, but not provided by underlying network. To design this \emph{repo} protocol, the following questions must be answered first:

\subsubsection{what is the storage unit}

The basic storage unit is data object. A data object is not just limited to one NDN data packet but defined by application level. The name of data object adopts the naming convention of NDN data packet except that, one data object could be segmented into multiple data packets. Although the data consumer can still access certain segmented packet of a data object, the basic operation unit is advised to be data object. The data object is immutable. If the data producer updates the data object, it should generate the data object of a new version.

Prefix is basic management unit, which means a set of name prefixes are registered to repo and repo will just regulates data objects under such prefixes.

\subsubsection{what functions does the \emph{repo} provide}

For a storage system, the basic operations are \emph{CRUD}. Currently, \emph{repo} offers data object retrieval, insertion, deletion and watch prefix. Data retrieval means \emph{repo} will response interests for data it holds. Data insertion and deletion are put and removal data objects of given names. Watch prefix function means that repo will keep sending interests of given interests for certain time.

\subsubsection{how to identify a repo}

One of the design goal of \emph{repo} is to operate on a designated \emph{repo}. There should be identities to distinguish \emph{repos}. In this design, the identity of \emph{repo} is \emph{repo} name. The format of \emph{repo} name conforms to the URL style of NDN name.

\subsubsection{how to encode command and response}

The command of \emph{repo} function is encoded in \emph{signed interest}\footnote{signed interest: http://redmine.named-data.net/projects/ndn-cxx/wiki/
SignedInterest}. Singed interest is a interest encoded with signature in the component of the name of the interest. This signature is signed by client who issues the command. Once the repo receives the \emph{repo} command, repo can validate the signature and to do specific operation according to the identity of the command. The basic structure of \emph{repo} command is that:

/$\langle repo-name\rangle$/$\langle function\rangle$/$\langle parameter\rangle$/$\langle timestamp\rangle$/
$\langle random-value\rangle$/$\langle SignatureInfo\rangle$/$\langle SignatureValue\rangle$

The last four components are necessary suffix of signed interests. \emph{Repo-name} is the name of the repo. \emph{function} is the name of the function, fox example, \emph{insert} for insert function. \emph{Parameter} is the parameter of the function. Multiple sub-components could be encoded in this parameter name component. For example, the parameter can carry the name of the data inserted for insert function.

Response is the responding data packet for command singed interest. Multiple information could be encoded in the subsection content of the data packet. For example, statuscode can denotes the status for repo to handle the command.

\subsubsection{how to transport command to designated \emph{repo}}

The transport mechanism of NDN is ``pull'', different from ``push'' of TCP/IP. Interest will not be forwarded to a designated host. However, command should be forwarded to the operated \emph{repo}. According to above, each \emph{repo} has its name and the prefix of command interest is \emph{repo-name}. If the \emph{repo-name} is unique and the routing path is generated correctly, the interest would be forwarded to the designated \emph{repo}. If multiple \emph{repos} share the same name, the forwarding results would be influenced by underlying forwarding strategy.

\subsubsection{how to secure the \emph{repo}}

In NDN, data packet carries the signature for validation. \emph{repo} command is encoded in signed interest also carries the signature. Thus, the similar security policy could be adopted to validate the \emph{repo} command. In addition, identity of command issuer is also encoded in the command interest. \emph{repo} can decide whether this identity has the access to the functions. Access control can be made. In \emph{repo} protocol design, policy of trust and access control are not limited. The users can make their own strategies but just conform to the format \emph{repo} commands.

\subsubsection{how to design functions of \emph{repo}}

The design goal or \emph{repo} functions is to make sure the status is visible to the client. This visibility is the basis of process control. For example, function insertion is to put data objects into repo. Client will send the command interest first to ask \emph{repo} to fetch the data with the name in the command parameter. If the client is also data producer, it will wait for the incoming interests. However, client will not know data is successfully put into the \emph{repo}. Client will wait until timeout, and cannot decide what to do. Thus, for each function of \emph{repo}, there should be status check process to status of the process, including whether the data object is successfully put or removed. In \emph{repo} design, different status check command comes with each function. This command will fetch the status and progres during the function operation.

\subsection{Data Retrieval}

As a storage application in NDN, \emph{repo} can provide data objects that it holds. Different from IP network, where data retrieval can only be achieved when requester have the knowledge about data location, the data request process in \emph{repo} goes into a more direct way. To retrieve a data from \emph{repo}, there is no need for the requesters to know which \emph{repo} owns the data it needs. During \emph{repo} initiation, data request prefixes (which is the data name in most cases) are registered into NDN forwarding daemon (NFD)\footnote{NFD: http://redmine.named-data.net/projects/nfd/wiki}. \emph{Repo} keeps listening to its registered prefixes and only these registered requests will be forwarded by NFD to this \emph{repo}. To make sure efficiency of data retrieval, only the names of data that \emph{repo} current maintains should be registered. Once the \emph{repo} receives a request, it will response the data if it currently owns, otherwise, it will ignore the request.

\subsection{Data Insertion}

\emph{Repo} data insertion represents the process that inserting a specific data object into Repo database, either obtained from local or remote node in the network. Data insertion is necessary since it is the only mechanism to append new data into \emph{repo}.

Data insertion process starts from receiving a signed interest. Instead of providing the data directly to Repo (since in most cases, the client who sends the signed interest does not have the target data), the signed interest gives instruction on how to get the data. The signed interest should specify the data name that will be inserted, and Repo can use this name to request the data in network by sending normal interest.

To insert data efficiently and correctly, \emph{repo} provides three different types of insertion: single insertion, insertion with selectors and segmented insertion. The signed interest will give Repo instruction that which insertion mechanism should be adopted.

\begin{enumerate}
\item Single insertion can be used to fetch the object with small size, which can be encoded into one single data packet. Only one interest can bring back the whole object and \emph{repo} store the data after data arrives.
\item Insertion with selectors will be triggered when signed interest contain name selectors. Selectors are used to narrow down the scope that the name prefix refers to. The normal interest that \emph{repo} sends to fetch the data should contain the exact same selectors specified in singed interest.
\item Segmented insertion deals with large object whose size exceeds the upper bound of one single data packet. Repo will send multiple interests with same name prefix but incremental sequence number to fetch the segments of large object. To avoid multiple segmented data return back in out of order, there could be a congestion control mechanism, which uses pipeline to request and receives a certain number (N) of continues segmented data in a time. Once a segment data with sequence number M comes back, Repo will send a new interest to fetch the next segment whose sequence number should be M+N.
    
    Besides, for segmented insertion, start block id, which represents the first segment \emph{repo} needs to fetch, and end block id, which represent the last segment \emph{repo} should fetch, can be specified in signed interest to only insert parts of segmented large object to \emph{repo}. It is possible that start block id and end block id are not specified at the same time. The default start block id is zero, which represents \emph{repo} should fetch from the first existed segment. However, there is no default end block id since the total number of segments for different object varies with the object size. And a problem arises due to \emph{repo} does not have the knowledge about which segment is the last one. If the end block id is not specified, \emph{repo} will continuously fetch the next segments even if they are not existed. There are two methods working together to avoid this problem. 1) The data producer who responds the data can specify the final block id (the last segment number that existed) in every segment data. When \emph{repo} receives the segments with the final block id specified, it will know which sequence number is the last one and when should stop insertion process. 2) \emph{repo} sets up a timer when end block id is not specified. When timeout happens, \emph{repo} will stop fetching data to store and end insert process. If the segment data with the final block id arrives, this timeout timer will be dismissed.
\end{enumerate}

Retransmission mechanism is used to handle data packet loss. The timeout for an interest may infer data packet loss. Whenever there is a timeout happen, \emph{repo} retransmit the interest. \emph{Repo} will stop the insertion process until the total number of retransmission reach a certain value.

To track the insertion process, a check command is used to get the current \emph{repo} status. The check command is implemented by signed interest and automatically issued by the same client who sends the insertion signed interest. It can bring back a status code which gives the client a hint that whether the insertion is successfully done.

 A tool named \emph{ndnputfile} is implemented to generate and send insertion signed interest. This tool also serves as data producer to those \emph{repos} who receive this signed interest. It helps those \emph{repos} to insert specific data either maintained locally or generated from stdin. Users can choose one from the three types of insertion and specify other parameters mentioned above in command line.

\subsection{Data Deletion}

To manage data more easily and remove unnecessary data to improve overall performance, \emph{repo} support the functionality of deletion. The process of deletion is similar to insertion, starting from receiving deletion signed interests. Instead of sending normal interest to fetch data, during deletion, \emph{repo} only needs to find the data whose name is specified in signed interest and remove it
from local database and index.

Similar to insertion, there are also three types of deletion: single deletion, deletion with selectors and segmented deletion. The process of these deletions are introduced as follow:

\begin{enumerate}
\item Single deletion is used to delete data packets (may not be a single one) stored in \emph{repo}. It will delete all the data that satisfy the name specified in signed interest (the name is the prefix of the data name).
\item Deletion with selectors is triggered when signed interest contain name selectors. \emph{repo} will remove all the data that match the selectors.
\item Segmented deletion is used to delete multiple segment data under same name prefix but have different sequence numbers. Start block id (optional, default is zero) and end block id (required) can be used to select the segments that should be deleted.
\end{enumerate}

From above, we can see that deletion process removes as much data as possible when receives a signed interest. It can also remove all the data under certain name space by only specifying the name space prefix in signed interest. In this way, \emph{repo} can achieve efficient deletion by issuing minimum number of interests. Besides, there are two methods to remove exact one data packet in \emph{repo}: 1) specify the full name in signed interest (since \emph{repo} use the longest-prefix match algorithm to find data) or 2) set the selectors to locate the exact target data.

There is also a similar check command used to check the status of \emph{repo} deletion. The check command will be sent automatically by the client after a deletion signed interest is issued, and this check command will bring back the status code which notify the client whether the data has been deleted or not.

The tool, ndndeletefile, is implemented to issue the deletion signed interest. The deletion type and parameters can be set by users.

\subsection{watch prefix}

With the hierarchical name system and selectors mechanism, \emph{repo} can support a new insertion algorithm, watched prefix. Compared with normal insertion process, which can only instruct \emph{repo} to insert data that have already existed, watched prefix allows Repo to keep watching certain data prefix, and insert data that may not be existed at that point and that may be generated in the future. Using watched prefix mechanism, \emph{repo} are no longer limited to insert only existed data. Instead, it can obtain some data that are generated after the moment it receives the command, by periodically watching those data in the network. Normal insertion process cannot support this function since \emph{repo} will try to fetch the data immediately when it receives the signed interest, and insertion will fails if there is no such data in network. For normal insertion, inserting a data is an instant process, but for watched prefix, it is a lasting and continues process.

When \emph{repo} receives the watched prefix signed interest, it will start this process. The signed interest need to specify the prefix that \emph{repo} should watch. \emph{repo} automatically and periodically generates the interests with that prefix and sends them into the network. If there are data returning back, \emph{repo} will update the interests by setting the exclude selector to exclude received data. In this way, Repo can avoid receiving duplicated data multiple times. If there is no data returning until timeout, \emph{repo} will resend the same interest.

Besides, watched prefix process can be stopped in three different ways: 1) receive stop command, which is also a signed interest, 2) process timeout or 3) the total number of interests have been sent exceeds a certain value. 1) provides a method to manually terminate this process and both 2) and 3) can be used to stop the process automatically. The check command can also be used in watched prefix process to check whether the process is still running.

Furthermore, there is no limitation for how many prefixes a Repo can watch at the same time. It is entirely possible that a Repo watches multiple prefix simultaneously. Every watched prefix process can work independently without interference.
We also implement a tool, ndnwatchprefix, which can generate the watched prefix start command, stop command and check command based on the input information. The prefix for each command is required. Users can also specify the process timeout value and the maximum number of interests the watched prefix process can send.

\section{implementation of repo -- repo-ng} \label{section-implementation}
\emph{Repo-ng} (NDN repo of new generation) is an implementation of NDN persistent in-network storage conforming to \emph{NDN Repo} protocol. It uses ndn-cxx as NDN client library and database Sqlite3 as underlying data storage.

\subsection{Repo Storage Design}

The design of data packet storage of \emph{repo-ng} consists of index and storage components. Index is sorted associative container for Name stored in memory and storage is consistent storage for read, write and delete of data packet.

Underlying storage offers structured API for retrieval, adding and removal of data packets. Sqlite3 is chosen to be the underlying storage for its well cross-platform, self-contained, serverless, and zero-configuration properties.

Index offers fast query of interest in memory. When an interest comes to \emph{repo}, \emph{repo} will query the index with the name and selector to select the id of data packet which is consistent with that of underlying storage. The structure of one entry of index is (Id, Name, KeyLocatorHash). Name is the name of data packet. KeyLocatorHash is the hash of key locator of data packet.

In current \emph{repo-ng} design, the data structure of repo-index is skip list. Although skip list, Btree and other balanced tree all have O(n) query and insertion complexity, skip list has very low inherent constant-factor overheads. Besides, operations of skiplist are much easier to implement. \cite{pugh1990skip}

\subsection{Tust Model and Access Control}

\subsubsection{Tust Model}

Command Interest and Data Packet can both be validated by \emph{repo-ng}. Whether or not validating is an option that can be configured. However, validation of command interest is highly recommended. In \emph{repo-ng} implementation, ``validator-config'' which is part of NDN basic library ``ndn-cxx'' is used as validator. The configuration file follows the format of validconf. \cite{validconf}

\subsubsection{Access Control}
hen a command interest is coming, each interest would be matched in Access Control List (ACL) to interest one by one, the best-matched rule is found. The design of ACL is referred to the format of validconf \cite{validconf}. Filter and checker are also defined in one rule. First, filter will check the prefix of the command interest to match the rule. Then filter will check the authorization of this filter according to the checker. The conceptual ACL is like Table \ref{ACL}:

\begin{table}[htbp]
\centering
\caption{Access Control List}
\label{ACL}
\begin{tabular}{ | c | c | c | c | }
    \hline
    repo-prefix & data-prefix & write-access & delete access \\ \hline
    /repo/example/1 & /data/example/1 & 1 & 0 \\ \hline
    /repo/example/1 & /data/example/2 & 0 & 1 \\ \hline
    /repo/example/2 & /data/example/3 & 1 & 1 \\ \hline
\end{tabular}
\end{table}

Repo-prefix is the repo-prefix in \emph{repo} command interest. Data-prefix is that in RepoCommandParameter. Write and delete access defines the access of insertion and deletion.

\subsection{Congestion Control}
In process of insertion, after \emph{repo} receiving the insertion command for segmented data, it will send multiple interests for data packets. If these interests are sent in a burst, the round trip time will be large for some interests and it is difficult to set interest lifetime. So congestion control is necessary.

A basic credit based congestion control is implemented for insertion command. There is credit number to count the waiting interest request. Once an interest is sent, the credit minus one and if an interest is satisfied by one data. If credit is equal or less than 0, the \emph{repo} stops sending interests. The original credit number could be adjusted according to the count of interest on the wire.

In addition, if an interest is timeout. The same interest could be retransmitted for certain times that can be configured. The retransmitted interest will not subtract the credit.

\section{Evaluation} \label{section-evaluation}

To fully understand the characteristics and benefits of \emph{repo} storage mechanism, in this section, the performance of \emph{repo} is evaluated. The evaluation can be separated into four parts: 1) performance of local storage, 2) network access, 3) factors influencing data retrieval and 4) fairness among multiple \emph{repos}.

For the first two parts, we test the \emph{repo} performance by running \emph{repos} in real situation. Tests are conducted on the following hardware platform: HP Z220 work station with 3.4 GHz Intel Core i7 processor of 8 cores, 16GB memory, 2T hard drive with 7200rpm, Ethernet Card 1000Mbps.  Lenovo notebook with 1.74GHz Intel Core i3 processor of 2 cores, 2GB memory, 500G hard drive with 7200rpm, Ethernet Card 1000Mbps.

For the last two parts, we conduct simulation-based experiments to evaluate repo performance under different conditions. NS3\footnote{ns-3: a discrete-event network simulator for Internet systems, http://www.nsnam.org} with ndnSim \cite{afanasyev2012ndnsim} model is used to simulate multiple scenarios.

\subsection{local access}
In this part, we evaluate the performance of \emph{repo} storage system, which includes database and index. The reason why we care about the storage performance is because the storage system of \emph{repo} is different from others. The speed of data fetching, insertion and deletion of data packets are measured. In addition, comparison between \emph{repo} over NFD platform and \emph{ccnr} over ccnd platform are conducted.

\emph{Repo} and \emph{ccnr} both supports retrieving and inserting data packets, but \emph{repo} also supports deleting data packets from repo. In this section, all the operations are tested on one hosts. The data packets are directly generated in memory to avoid hard drive I/O overhead. Access control is not configured. The following are scenarios of tests:

\begin{enumerate}[a]
\item Put $10^3$, $10^4$, $10^5$, $10^6$ data packets with 1200 bytes of data content into a clean \emph{repo-ng}. Validation is off.
\item Put $10^3$ data packets with $10^3$. Insert commands into a clean \emph{repo-ng} with and without validation.
\item Retrieve $10^3$, $10^4$, $10^5$, $10^6$ data packets from \emph{repo-ng} which carries $10^3$, $10^4$, $10^5$, $10^6$ data packets of 1200Bits.
\item Remove $10^3$, $10^4$, $10^5$, $10^6$ data packets from \emph{repo-ng} which carries $10^3$, $10^4$, $10^5$, $10^6$ data packets of 1200Bits. Validation is off.
\item Rebuild index from $10^3$, $10^4$, $10^5$, $10^6$ data packets from database file.
\item Put $10^3$, $10^4$, $10^5$, $10^6$ data packets with 1200 bytes of data content into a clean \emph{ccnr}.
\item Retrieve $10^3$, $10^4$, $10^5$, $10^6$ data packets from \emph{ccnr} which carries $10^3$, $10^4$, $10^5$, $10^6$ data packets of 1200Bits.
\end{enumerate}

Table \ref{local-repo} shows the results of \emph{repo-ng} speed from case a to case e. The unit of result is MBps. ``put -s'' and ``put-s-v'' mean insert commands into a clean \emph{repo-ng} with and without validation

\begin{table}[htbp]
\centering
\caption{Local Access of Repo-ng}
\label{local-repo}
\begin{tabular}{ | c | c | c | c | c | c | c | }
    \hline
           & put & get & remove & rebuild & put-s & put-s-v \\ \hline
    $10^3$ & 0.692 & 16.881 & 15.584 & 120 & 0.041 & 0.038  \\ \hline
    $10^4$ & 0.715 & 16.585 & 25.974 & 153.846 & & \\ \hline
    $10^5$ & 0.719 & 16.634 & 29.843 & 158.521 & & \\ \hline
    $10^6$ & 0.713 & 12.918 & 26.266 & 95.610 & &\\ \hline
\end{tabular}
\end{table}

Table \ref{local-ccnr}  shows the the results of \emph{ccnr} speed from case f to case g

\begin{table}[htbp]
\centering
\caption{Local Access of Ccnr}
\label{local-ccnr}
\begin{tabular}{ | c | c | c | }
    \hline
           & put & get \\ \hline
    $10^3$ & 1.336 & 0.869 \\ \hline
    $10^4$ & 8.778 & 4.180 \\ \hline
    $10^5$ & 24.995 & 13.258 \\ \hline
    $10^6$ & 28.323 & 18.266 \\ \hline
\end{tabular}
\end{table}

Table \ref{local-repo} and \ref{local-ccnr} shows the comparison of throughput between \emph{repo-ng} and \emph{ccnr}. The speed of insertion of \emph{ccnr} is far more higher than that of \emph{repo-ng}. One of main reason is the underlying storage of these systems. The storage of \emph{ccnr} is rather simple by appending new content objects in the \emph{ccnr} file. In \emph{repo-ng} design, it would support more complex query (selector) of data packets, so it adopts structured storage database. The process of database insertion is far more complex than just appending a file. The speed of fetching is comparable between \emph{repo-ng} and \emph{ccnr}. As we notice that, the speed of fetching drops a little as the mounts of packet grows. The index of \emph{repo-ng} is based on skiplist. It will took more time when index gets larger. Besides, query of database will get slower when data gets larger.

Table \ref{local-repo} also shows the speed of possessing multiple commands. It takes 29.468s and 31.619s to possess 1000 insetion commands with and without validation. The time cost of insertion data packets into database could be dismissed. It takes about 30ms to possess a command locally. In our validation case, the \emph{repo-ng} directly validated commands and data packets using the local certificate file. As we can see, the time cost of validation is rather low.

\subsection{network access}

In this section, we test the performance of \emph{repo} protocols in real network, in order to understand how efficiently \emph{repo} can work in real environment. We run two \emph{repos} on the work stations mentioned above separately. The one running on Note-book serves as client and the other one running on HP Z220 works as a server. The two work stations are directly connected. The differences between this section and the previous one are: 1) the signed interests are generated by the server and transmitted to the client via network, instead of both generated and applied locally, and 2) client needs to fetch the data stored in server when the signed interests are specified as retrieval or insertion, instead of obtaining the data generated from memory.

Since data transmissions are fast between the client and server, to quantitatively evaluate the performance of \emph{repo} protocols, we measure the time of the whole process, started from the generation of signed interest and finished when \emph{repo} correctly handles the data.

We use the same scenarios (case a to case d mentioned in local access section) to test the performance with network connection. The results are also measured by throughput (Mbps).

\begin{table}[htbp]
\centering
\caption{Network Access of Repo-ng}
\label{network-repo}
\begin{tabular}{ | c | c | c | c | c | c | }
    \hline
           & put & get & remove & put-s & put-s-v \\ \hline
    $10^3$ & 0.396 & 2.230 & 8.053 & 0.033 & 0.032  \\ \hline
    $10^4$ & 0.423 & 3.490 & 21.898 & & \\ \hline
    $10^5$ & 0.0.424 & 3.476 & 24.964 & & \\ \hline
\end{tabular}
\end{table}

Compared with local access, the speed drops due to the network latency between the two hosts. However, we can see a great speed drop on retrieving data packets. Different resources of client and \emph{repo} are tested to track down the performance bottleneck. We see that the  load of client CPU usage is up to 70\%. This is caused by the process of inserting data. When inserting data into the \emph{repo}, \emph{repo} will issues multiple interests for each data packet. The possessing of these interests would be a great load for the low hardware setting notebook.

\emph{Repo-ng} is just one-thread process for now. There is some inefficiency to serve multiple client. For data retrieval, the cost of possessing each interest is little, so this one-thread design would not impact much. However, insertion or removal of large mount of data packets would block the \emph{repo-ng} process for a certain time, while \emph{repo-ng} cannot possess other data handle requests during this time.  Therefore, a time-cost process may cost congestion. Multi-thread or other types of asynchronous design should be adopted in future.

\subsection{Factors influencing data retrieval}
The reason why we care about data retrieval is because it is the operation that is most necessary and frequently conducted by repo (used in data insertion, retrieval and watched prefix). Due to the intrinsic properties of fetching mechanism, multiple rounds of sending interests and receiving data is necessary in request data (e.g. signed interest, segment insertion and retransmission). In light of this, the network conditions, including network delay and packet loss, may play a role in repo performance. Besides, the efficiency of data retrieval also depends on cost spent on database accessing (match the correct data in the database). Since the database processing is fully evaluated in Local storage section (section A), in this section, we will mainly focus on illustrating the performance under the influence of multiple network conditions.

\subsubsection{Network Latency}

The first network condition we will discuss is network latency. The network latency influences the transmission time of both interests and data. Imagine, if \emph{repo} works in a complex network with significant network latency, it is entirely possible that the interests expire before reaching to the node that maintains the target data. We evaluate how the total throughput of the network varies with the network latency. The throughput is measured by Mbps.

To avoid unnecessary overhead, we use the simplest topology to test the throughput (Fig. 1) (there should be label here). In the simulation, the data rate of each link is 100 Mbps and drop-tail queue has capacity of 20000 packets, with 0 drop rate. The client repo sends 1000 data request to the server via a NDN router to request the data. After receiving the request, the server search the requested data in local database and return the data. All the requests can be satisfied by the server. We measure the time from sending the first request to receiving all the data packets.

From the (Fig.2 there should be label) we can see that the throughput decreases with the network latency increases. To fully understand the influence of network latency, we reveal the performance with and without database processing separately. The blue curve represents there is no delay for database accessing and the throughput is only calculated from data transmission. And the yellow curve reveals the throughput of whole process, including database accessing and transmission. We can see in the network with small latency (lower than 20ms), the database accessing is the bottle neck for data retrieval. However, with the increasing of network latency, the cost of searching data in database becomes trivial, and the low throughput mainly result from the long delay of data transmission.

\subsubsection{packet loss}

Next, we want to talk about how packet loss lays the impact on data retrieval for \emph{repo}. Due to the properties of NDN routing, it is only when the interest timeout and no data come back can \emph{repo} know that there is something wrong. In this section, we assume that all the requested data exist in the network and the timeout only results from the packet drop, either interest or data.

We simulate packet drop under different drop rates. The Rate Error Model in NS3 is applied to all the links in the network with same drop rate in each simulation. For \emph{repo}, whenever there is an interest timeout, it will re-issue the same interest immediately. For a certain interest, the maximum retransmission number is 4. If there is no data back after four times retransmission, repo aborts this request. We calculate the distribution of retransmission times under different packet drop rate to see how the packet loss influence the data retrieval. ��/�� is used to represent the aborted requests.

From the (Fig. 3), we can see that under certain packet drop rate, most requests only need to be retransmitted once, however, there are still a few requests that need to be retransmitted multiple times since it is possible that packet loss happens during the retransmission.  If the packets of same requests, either interests or data, drop multiple times, the waiting time should at least be interest lifetime multiple the number of retransmission. The situation goes worse in segment insertion, where the pipeline mechanism force \emph{repo} to fetch segment data in order. If one is missing, \emph{repo} will retransmit the corresponding interest and other segment data cannot be fetching, which leads to the postponement of the whole insertion process.

Other network conditions, such as congestions and loop, are also fully evaluated. However, their contributions to the performance of data retrieval are not as significant as the two factors mentioned above. It can be explained either by some mechanisms designed to deal with them (\emph{repo} congestion control mechanisms), or by the intrinsic properties of NDN routing. Therefore due to the limited space, we will not present the evaluation result here.

    

\section{discussion} \label{section-discussion}
\subsection{discussion about the above evaluation}

\subsection{What are the advantages of this design}

\begin{itemize}
\item Compared with ccnr. We have remote operations, removing functions, trust management
\item compared with general network storage system. Repo stores application level object. network ready data packet. avoid mapping between storage data and network data.
\end{itemize}

\section{conclusion and future work} \label{section-conclusion}
conclusion points:
\begin{itemize}
\item \emph{NDN repo} implements concept of application level framing. Repo function process, transport control, security policy are directly applied in application level.
\item Compared to \emph{ccnr}, \emph{NDN repo} is a more functional application for other NDN application to store data. Although some performance would be tradeoff for more functions.
\end{itemize}

Future work: multiple repo collaboration / repo sync

\bibliographystyle{IEEEtran}
\bibliography{IEEEabrv,repo}
% that's all folks
\end{document}


