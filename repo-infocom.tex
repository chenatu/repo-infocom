
%% bare_conf.tex
%% V1.3
%% 2007/01/11
%% by Michael Shell
%% See:
%% http://www.michaelshell.org/
%% for current contact information.
%%
%% This is a skeleton file demonstrating the use of IEEEtran.cls
%% (requires IEEEtran.cls version 1.7 or later) with an IEEE conference paper.
%%
%% Support sites:
%% http://www.michaelshell.org/tex/ieeetran/
%% http://www.ctan.org/tex-archive/macros/latex/contrib/IEEEtran/
%% and
%% http://www.ieee.org/

%%*************************************************************************
%% Legal Notice:
%% This code is offered as-is without any warranty either expressed or
%% implied; without even the implied warranty of MERCHANTABILITY or
%% FITNESS FOR A PARTICULAR PURPOSE!
%% User assumes all risk.
%% In no event shall IEEE or any contributor to this code be liable for
%% any damages or losses, including, but not limited to, incidental,
%% consequential, or any other damages, resulting from the use or misuse
%% of any information contained here.
%%
%% All comments are the opinions of their respective authors and are not
%% necessarily endorsed by the IEEE.
%%
%% This work is distributed under the LaTeX Project Public License (LPPL)
%% ( http://www.latex-project.org/ ) version 1.3, and may be freely used,
%% distributed and modified. A copy of the LPPL, version 1.3, is included
%% in the base LaTeX documentation of all distributions of LaTeX released
%% 2003/12/01 or later.
%% Retain all contribution notices and credits.
%% ** Modified files should be clearly indicated as such, including  **
%% ** renaming them and changing author support contact information. **
%%
%% File list of work: IEEEtran.cls, IEEEtran_HOWTO.pdf, bare_adv.tex,
%%                    bare_conf.tex, bare_jrnl.tex, bare_jrnl_compsoc.tex
%%*************************************************************************

% *** Authors should verify (and, if needed, correct) their LaTeX system  ***
% *** with the testflow diagnostic prior to trusting their LaTeX platform ***
% *** with production work. IEEE's font choices can trigger bugs that do  ***
% *** not appear when using other class files.                            ***
% The testflow support page is at:
% http://www.michaelshell.org/tex/testflow/



% Note that the a4paper option is mainly intended so that authors in
% countries using A4 can easily print to A4 and see how their papers will
% look in print - the typesetting of the document will not typically be
% affected with changes in paper size (but the bottom and side margins will).
% Use the testflow package mentioned above to verify correct handling of
% both paper sizes by the user's LaTeX system.
%
% Also note that the "draftcls" or "draftclsnofoot", not "draft", option
% should be used if it is desired that the figures are to be displayed in
% draft mode.
%
\documentclass[conference]{IEEEtran}

\usepackage{url, fancyvrb, framed, multirow, tabularx, graphicx, epstopdf, enumerate, array, cite, algorithmic, fixltx2e}

\usepackage[cmex10]{amsmath}
%\usepackage{breqn}

% correct bad hyphenation here
\hyphenation{op-tical net-works semi-conduc-tor}

\begin{document}
%
% paper title
% can use linebreaks \\ within to get better formatting as desired
\title{NDN Repo: An NDN Persistent Storage Model}


% author names and affiliations
% use a multiple column layout for up to three different
% affiliations
%\author{\IEEEauthorblockN{Michael Shell}
%\IEEEauthorblockA{School of Electrical and\\Computer Engineering\\
%Georgia Institute of Technology\\
%Atlanta, Georgia 30332--0250\\
%Email: http://www.michaelshell.org/contact.html}
%\and
%\IEEEauthorblockN{Homer Simpson}
%\IEEEauthorblockA{Twentieth Century Fox\\
%Springfield, USA\\
%Email: homer@thesimpsons.com}
%\and
%\IEEEauthorblockN{James Kirk\\ and Montgomery Scott}
%\IEEEauthorblockA{Starfleet Academy\\
%San Francisco, California 96678-2391\\
%Telephone: (800) 555--1212\\
%Fax: (888) 555--1212}}

% conference papers do not typically use \thanks and this command
% is locked out in conference mode. If really needed, such as for
% the acknowledgment of grants, issue a \IEEEoverridecommandlockouts
% after \documentclass

% for over three affiliations, or if they all won't fit within the width
% of the page, use this alternative format:
%
%\author{\IEEEauthorblockN{Michael Shell\IEEEauthorrefmark{1},
%Homer Simpson\IEEEauthorrefmark{2},
%James Kirk\IEEEauthorrefmark{3},
%Montgomery Scott\IEEEauthorrefmark{3} and
%Eldon Tyrell\IEEEauthorrefmark{4}}
%\IEEEauthorblockA{\IEEEauthorrefmark{1}School of Electrical and Computer Engineering\\
%Georgia Institute of Technology,
%Atlanta, Georgia 30332--0250\\ Email: see http://www.michaelshell.org/contact.html}
%\IEEEauthorblockA{\IEEEauthorrefmark{2}Twentieth Century Fox, Springfield, USA\\
%Email: homer@thesimpsons.com}
%\IEEEauthorblockA{\IEEEauthorrefmark{3}Starfleet Academy, San Francisco, California 96678-2391\\
%Telephone: (800) 555--1212, Fax: (888) 555--1212}
%\IEEEauthorblockA{\IEEEauthorrefmark{4}Tyrell Inc., 123 Replicant Street, Los Angeles, California 90210--4321}}




% use for special paper notices
%\IEEEspecialpapernotice{(Invited Paper)}




% make the title area
\maketitle


\begin{abstract}
%\boldmath

\end{abstract}
% IEEEtran.cls defaults to using nonbold math in the Abstract.
% This preserves the distinction between vectors and scalars. However,
% if the conference you are submitting to favors bold math in the abstract,
% then you can use LaTeX's standard command \boldmath at the very start
% of the abstract to achieve this. Many IEEE journals/conferences frown on
% math in the abstract anyway.

% no keywords




% For peer review papers, you can put extra information on the cover
% page as needed:
% \ifCLASSOPTIONpeerreview
% \begin{center} \bfseries EDICS Category: 3-BBND \end{center}
% \fi
%
% For peerreview papers, this IEEEtran command inserts a page break and
% creates the second title. It will be ignored for other modes.
\IEEEpeerreviewmaketitle

\section{Introduction}

Data distribution is the largest network stream in current Internet. Many web applications like video or image distribution adopts content delivery network (CDN) for quick access of network data. Besides, more and more data are stored in network storage such as Dropbox, Google Drive and Amazon S3. Network and backend storage are commonly separately designed in current TCP/IP network architecture and there is always network packet and storage data presentation transforming process. In this paper, an in-network storage model Named Data Networking repository (\emph{NDN repo}) is proposed based on NDN network. In NDN, network packet is directly identified by the name of the data, not the source or destination identification. Thanks to this mechanism, \emph{NDN repo} directly storages network and application ready data, which realizes the concept of application level framing. \cite{clark1990architectural}

An \emph{NDN repo} is a set of storage application over NDN network managed by single party. It is upon application level without tweaking the NDN protocol. Compared with \emph{Content Store} (\emph{CS}) which provides data packets in-network cache, \emph{NDN repo} provides persistent storage for data objects. The storage of unit is data object and the management scope is based on NDN name prefix. An NDN repo should response interests with managed name prefix with data packets. Besides, NDN repo also offers data object insertion and deletion, fetching data with assigned name prefix and status check of such progresses. An \emph{NDN repo} should conform to the \emph{NDN repo} protocol, which specifies the semantics of NDN packet for repo interaction and basic processes of each functions, such as insertion and deletion. Repo protocol does not limit packet transport control or any security policy.

In this paper, the design of \emph{NDN repo} protocol is demonstrated. The major design goals are: security for remote operation, management and control based on namespace, reliable operation of large data object which can not be framed in a single data packet. Different special cases are detialedly discussed. An initial \emph{NDN repo} implementation \emph{repo-ng} is also demonstrated. It conforms to the protocol and provides interfaces to configure security policy. The boundary of network and storage is broken that data object could be directly used for application and network transportation. Besides, \emph{NDN repo} protocol can be applied for NDN application using data storage service.

The rest of this paper is organized as follows. Section \ref{section-background} introduce the background of \emph{NDN repo}.  Section \ref{section-design} illustrates the design goals of \emph{NDN repo} and how it works. Section \ref{section-implementation} demonstrates the example implementation of \emph{NDN repo} -- \emph{repo-ng}. Section \ref{section-evaluation} evaluates the performance and functionality of \emph{repo-ng}. Section \ref{section-discussion} discusses the evaluation results and tradeoff in \emph{NDN repo} design. Section \ref{section-conclusion} concludes the paper and addresses the future work.

\section{Background} \label{section-background}
\subsection{Named Data Networking}
Named Data Networking (NDN) \cite{zhang2010named} is a data-oriented network architecture which replaces IP with names of data packets as the narrow waist of networking. The essential evolution of NDN is the change of network behaviors from delivering data to a certain destination to fetching data with a given name. \cite{zhang2010named} Because of this change, \emph{interest} and \emph{data object} are imported as network packets for fetching and responding data of given name. \emph{Interest} is the request of network packets containing prefix of names and other constraints. Data packet contains the content of data and the digital signature signed by data producer.

In-network storage means device at network level can not only cache network packet temporarily, but also stores the data packet for local or remote application using directly. NDN makes in-network storage possible because of following reasons: Network packet is identified by source and destination host addresses in IP network, while name of NDN packet is irrelevant with physical endpoints. Any host in NDN network carrying data of given names can response to the \emph{interest}, but hosts besides source and destination cannot retrieve in-network IP packet. Another concern is the privacy of in-network data. Signature in data packet is to resolve authentication and confidentiality of data. \emph{Content Store (CS)} is cache of data packet in NDN router model and it is within network layer.

In \cite{clark1990architectural}, TCP/IP and other layered network architecture are revisited. Data presentation is major cost in network data processing. This presentation process should not be limited to the current network function, but should adjust to certain application wants. Concept of application level framing is proposed to move presentation and transport control from network to application level. This principle is fully adopted in \emph{NDN repo} design.

\subsection{ccnr}
 \emph{Ccnr} is a subset software of CCNx project\footnote{CCNx: http://www.ccnx.org/what-is-ccn/}. CCN and NDN have the same origin and their architecture is similar. \emph{Cnnr} is a repository that preserving CCN network packet. It supports remote data fetching using interests and local data dump. The functions of \emph{ccnr} is limited. User cannot issue data insertion command remotely and command cannot be verified to apply security policy. This repository protocol is not sufficient for application needs.

\section{Design of NDN Repo protocol} \label{section-design}

\subsection{overall design of repo protocol}

The goal of this section is to demonstrate protocol and discuss concerns on potential failure cases.

NDN Repo protocol is specification of NDN network packet and process to operate \emph{NDN repo}. Controls of the network transportation such as flow control, access control and so on are all defined in \emph{repo} application level, but not provided by underlying network. To design this \emph{repo} protocol, the following questions must be answered first:

\subsubsection{what is the storage unit}

The basic storage unit is data object. A data object is not just limited to one NDN data packet but defined by application level. The name of data object adopts the naming convention of NDN data packet except that, one data object could be segmented into multiple data packets. Although the data consumer can still access certain segmented packet of a data object, the basic operation unit is advised to be data object. The data object is immutable. If the data producer updates the data object, it should generate the data object of a new version.

Prefix is basic management unit, which means a set of name prefixes are registered to repo and repo will just regulates data objects under such prefixes.

\subsubsection{what functions does the \emph{repo} provide}

For a storage system, the basic operations are \emph{CRUD}. Currently, \emph{repo} offers data object retrieval, insertion, deletion and watch prefix. Data retrieval means \emph{repo} will response interests for data it holds. Data insertion and deletion are put and removal data objects of given names. Watch prefix function means that repo will keep sending interests of given interests for certain time.

\subsubsection{how to identify a repo}

One of the design goal of \emph{repo} is to operate on a designated \emph{repo}. There should be identities to distinguish \emph{repos}. In this design, the identity of \emph{repo} is \emph{repo} name. The format of \emph{repo} name conforms to the URL style of NDN name.

\subsubsection{how to encode command and response}

The command of \emph{repo} function is encoded in \emph{signed interest}\footnote{signed interest: http://redmine.named-data.net/projects/ndn-cxx/wiki/
SignedInterest}. Singed interest is a interest encoded with signature in the component of the name of the interest. This signature is signed by client who issues the command. Once the repo receives the \emph{repo} command, repo can validate the signature and to do specific operation according to the identity of the command. The basic structure of \emph{repo} command is that:

/$\langle repo-name\rangle$/$\langle function\rangle$/$\langle parameter\rangle$/$\langle timestamp\rangle$/
$\langle random-value\rangle$/$\langle SignatureInfo\rangle$/$\langle SignatureValue\rangle$

The last four components are necessary suffix of signed interests. \emph{Repo-name} is the name of the repo. \emph{function} is the name of the function, fox example, \emph{insert} for insert function. \emph{Parameter} is the parameter of the function. Multiple sub-components could be encoded in this parameter name component. For example, the parameter can carry the name of the data inserted for insert function.

Response is the responding data packet for command singed interest. Multiple information could be encoded in the subsection content of the data packet. For example, statuscode can denotes the status for repo to handle the command.

\subsubsection{how to transport command to designated \emph{repo}}

The transport mechanism of NDN is ``pull'', different from ``push'' of TCP/IP. Interest will not be forwarded to a designated host. However, command should be forwarded to the operated \emph{repo}. According to above, each \emph{repo} has its name and the prefix of command interest is \emph{repo-name}. If the \emph{repo-name} is unique and the routing path is generated correctly, the interest would be forwarded to the designated \emph{repo}. If multiple \emph{repos} share the same name, the forwarding results would be influenced by underlying forwarding strategy.

\subsubsection{how to secure the \emph{repo}}

In NDN, data packet carries the signature for validation. \emph{repo} command is encoded in signed interest also carries the signature. Thus, the similar security policy could be adopted to validate the \emph{repo} command. In addition, identity of command issuer is also encoded in the command interest. \emph{repo} can decide whether this identity has the access to the functions. Access control can be made. In \emph{repo} protocol design, policy of trust and access control are not limited. The users can make their own strategies but just conform to the format \emph{repo} commands.

\subsubsection{how to design functions of \emph{repo}}

The design goal or \emph{repo} functions is to make sure the status is visible to the client. This visibility is the basis of process control. For example, function insertion is to put data objects into repo. Client will send the command interest first to ask \emph{repo} to fetch the data with the name in the command parameter. If the client is also data producer, it will wait for the incoming interests. However, client will not know data is successfully put into the \emph{repo}. Client will wait until timeout, and cannot decide what to do. Thus, for each function of \emph{repo}, there should be status check process to status of the process, including whether the data object is successfully put or removed. In \emph{repo} design, different status check command comes with each function. This command will fetch the status and progres during the function operation.

\subsection{read}

\subsection{insert}
\subsubsection{process of insert}
how to handle single insert, single insert with selector and segmented insert

while response command before fetch interests

\subsubsection{how to handle packet loss of data}

\subsubsection{how to handle congestion control}
repo implementation should design its own congestion control.

\subsubsection{how to handle no end time out}
NoEndTimeout means when command have start block id and no end block id. repo will keep fetch data packets

\subsubsection{how to build application using insert command}
give ndnputfile as an example.

\subsection{delete}

\subsubsection{process of delete}
how to handle single delete, single delete with selector and segmented delete

\subsubsection{special selector in delete}
selector not to select one data but all the data conforming to the constraints

\subsubsection{delete command lifetime timeout}

\subsubsection{what to do if fail to delete some of the data packets}

\subsubsection{how to build application using delete command}
ndndeletefile

\subsection{watch prefix}

\subsubsection{process of watch prefix}

\subsubsection{what to do if data time-out}

\subsubsection{why use watch prefix command}

\section{implementation of repo -- repo-ng} \label{section-implementation}
\subsection{storage-design}
using database sqlite for back-end storage: better cross-platform, light-weight easy to implement

using skiplist as index. for better performance rebuild index whenever repo restarts

\subsection{trust model and access control}
using valdiator-conf

access control list designed referred to validator-conf design

\subsection{congestion control design}

\section{Evaluation} \label{section-evaluation}

\subsection{local access}
throughput compared to ccnr

\subsection{network access}
throughput of one-hop network

maybe simulation of large-scale network to test failure cases.

\section{discussion} \label{section-discussion}
\subsection{discussion about the above evaluation}

\subsection{What are the advantages of this design}

\begin{itemize}
\item Compared with ccnr. We have remote operations, removing functions, trust management
\item compared with general network storage system. Repo stores application level object. network read data packet. avoid mapping between storage data and network data.
\end{itemize}

\section{conclusion and future work} \label{section-conclusion}
conclusion points:
\begin{itemize}
\item \emph{NDN repo} implements concept of application level framing. Repo function process, transport control, security policy are directly applied in application level.
\item Compared to \emph{ccnr}, \emph{NDN repo} is a more functional application for other NDN application to store data. Although some performance would be tradeoff for more functions.
\end{itemize}

Future work: multiple repo collaboration / repo sync

\bibliographystyle{IEEEtran}
\bibliography{IEEEabrv,repo}
% that's all folks
\end{document}


