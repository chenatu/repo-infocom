
%% bare_conf.tex
%% V1.3
%% 2007/01/11
%% by Michael Shell
%% See:
%% http://www.michaelshell.org/
%% for current contact information.
%%
%% This is a skeleton file demonstrating the use of IEEEtran.cls
%% (requires IEEEtran.cls version 1.7 or later) with an IEEE conference paper.
%%
%% Support sites:
%% http://www.michaelshell.org/tex/ieeetran/
%% http://www.ctan.org/tex-archive/macros/latex/contrib/IEEEtran/
%% and
%% http://www.ieee.org/

%%*************************************************************************
%% Legal Notice:
%% This code is offered as-is without any warranty either expressed or
%% implied; without even the implied warranty of MERCHANTABILITY or
%% FITNESS FOR A PARTICULAR PURPOSE!
%% User assumes all risk.
%% In no event shall IEEE or any contributor to this code be liable for
%% any damages or losses, including, but not limited to, incidental,
%% consequential, or any other damages, resulting from the use or misuse
%% of any information contained here.
%%
%% All comments are the opinions of their respective authors and are not
%% necessarily endorsed by the IEEE.
%%
%% This work is distributed under the LaTeX Project Public License (LPPL)
%% ( http://www.latex-project.org/ ) version 1.3, and may be freely used,
%% distributed and modified. A copy of the LPPL, version 1.3, is included
%% in the base LaTeX documentation of all distributions of LaTeX released
%% 2003/12/01 or later.
%% Retain all contribution notices and credits.
%% ** Modified files should be clearly indicated as such, including  **
%% ** renaming them and changing author support contact information. **
%%
%% File list of work: IEEEtran.cls, IEEEtran_HOWTO.pdf, bare_adv.tex,
%%                    bare_conf.tex, bare_jrnl.tex, bare_jrnl_compsoc.tex
%%*************************************************************************

% *** Authors should verify (and, if needed, correct) their LaTeX system  ***
% *** with the testflow diagnostic prior to trusting their LaTeX platform ***
% *** with production work. IEEE's font choices can trigger bugs that do  ***
% *** not appear when using other class files.                            ***
% The testflow support page is at:
% http://www.michaelshell.org/tex/testflow/



% Note that the a4paper option is mainly intended so that authors in
% countries using A4 can easily print to A4 and see how their papers will
% look in print - the typesetting of the document will not typically be
% affected with changes in paper size (but the bottom and side margins will).
% Use the testflow package mentioned above to verify correct handling of
% both paper sizes by the user's LaTeX system.
%
% Also note that the "draftcls" or "draftclsnofoot", not "draft", option
% should be used if it is desired that the figures are to be displayed in
% draft mode.
%
\documentclass[conference]{IEEEtran}

\usepackage{url, fancyvrb, framed, multirow, tabularx, graphicx, epstopdf, enumerate, array, cite, algorithmic, fixltx2e}

\usepackage[cmex10]{amsmath}
%\usepackage{breqn}

% correct bad hyphenation here
\hyphenation{op-tical net-works semi-conduc-tor}

\begin{document}
%
% paper title
% can use linebreaks \\ within to get better formatting as desired
\title{NDN Repo: An NDN Persistent Storage Model}


% author names and affiliations
% use a multiple column layout for up to three different
% affiliations
%\author{\IEEEauthorblockN{Michael Shell}
%\IEEEauthorblockA{School of Electrical and\\Computer Engineering\\
%Georgia Institute of Technology\\
%Atlanta, Georgia 30332--0250\\
%Email: http://www.michaelshell.org/contact.html}
%\and
%\IEEEauthorblockN{Homer Simpson}
%\IEEEauthorblockA{Twentieth Century Fox\\
%Springfield, USA\\
%Email: homer@thesimpsons.com}
%\and
%\IEEEauthorblockN{James Kirk\\ and Montgomery Scott}
%\IEEEauthorblockA{Starfleet Academy\\
%San Francisco, California 96678-2391\\
%Telephone: (800) 555--1212\\
%Fax: (888) 555--1212}}

% conference papers do not typically use \thanks and this command
% is locked out in conference mode. If really needed, such as for
% the acknowledgment of grants, issue a \IEEEoverridecommandlockouts
% after \documentclass

% for over three affiliations, or if they all won't fit within the width
% of the page, use this alternative format:
%
%\author{\IEEEauthorblockN{Michael Shell\IEEEauthorrefmark{1},
%Homer Simpson\IEEEauthorrefmark{2},
%James Kirk\IEEEauthorrefmark{3},
%Montgomery Scott\IEEEauthorrefmark{3} and
%Eldon Tyrell\IEEEauthorrefmark{4}}
%\IEEEauthorblockA{\IEEEauthorrefmark{1}School of Electrical and Computer Engineering\\
%Georgia Institute of Technology,
%Atlanta, Georgia 30332--0250\\ Email: see http://www.michaelshell.org/contact.html}
%\IEEEauthorblockA{\IEEEauthorrefmark{2}Twentieth Century Fox, Springfield, USA\\
%Email: homer@thesimpsons.com}
%\IEEEauthorblockA{\IEEEauthorrefmark{3}Starfleet Academy, San Francisco, California 96678-2391\\
%Telephone: (800) 555--1212, Fax: (888) 555--1212}
%\IEEEauthorblockA{\IEEEauthorrefmark{4}Tyrell Inc., 123 Replicant Street, Los Angeles, California 90210--4321}}




% use for special paper notices
%\IEEEspecialpapernotice{(Invited Paper)}




% make the title area
\maketitle


\begin{abstract}
%\boldmath

\end{abstract}
% IEEEtran.cls defaults to using nonbold math in the Abstract.
% This preserves the distinction between vectors and scalars. However,
% if the conference you are submitting to favors bold math in the abstract,
% then you can use LaTeX's standard command \boldmath at the very start
% of the abstract to achieve this. Many IEEE journals/conferences frown on
% math in the abstract anyway.

% no keywords




% For peer review papers, you can put extra information on the cover
% page as needed:
% \ifCLASSOPTIONpeerreview
% \begin{center} \bfseries EDICS Category: 3-BBND \end{center}
% \fi
%
% For peerreview papers, this IEEEtran command inserts a page break and
% creates the second title. It will be ignored for other modes.
\IEEEpeerreviewmaketitle

\section{Introduction}
Basic introduction of NDN and Why Naming and signature of data makes in-network storage possible.

What is repo: storage device of data object managed by single party. Management Unit: prefix of data object. Provided functions: read, insert, delete, watch prefix.

Basic ideas of repo: Application Level Object. Repo is an attempt to implement this idea. What can repo benefit from Application Level Object idea.

\section{Background}
\subsection{ccnr}
ccnr. what it cannot do compared to repo

\subsection{Naming Convention of NDN}
Naming convention of NDN. Repo is to handle immutable data.

\subsection{Application Level Object}
talk something about paper ``Architectural  Considerations  for  a  New  Generation  of  Protocols ''

\section{Design of Repo}
Address what problem should repo deal with:
\begin{enumerate}
\item how to identify a repo
\item what to encode command and response
\item how to transport command to designated repo
\item how to construct trust policy and access control (based on signature info in signed interest)
\item how to build functional application using repo (talking about necessity of status check command in helping building apps using repo)
\end{enumerate}

Summary of points above. talk about a simple process of a repo command.

What are the advantages of this design.

\begin{itemize}
\item Compared to ccnr. We have remote operations, removing functions, trust management
\item compared to general network storage system. Repo stores application level object. network read data packet. avoid mapping between storage data and network data.
\end{itemize}

what functions have currently supported and what are the definitions.

\subsection{read}

\subsection{insert}
\subsubsection{process of insert}
how to handle single insert, single insert with selector and segmented insert

while response command before fetch interests

\subsubsection{how to handle packet loss of data}

\subsubsection{how to handle congestion control}
repo implementation should design its own congestion control.

\subsubsection{how to handle no end time out}
NoEndTimeout means when command have start block id and no end block id. repo will keep fetch data packets

\subsubsection{how to build application using insert command}
give ndnputfile as an example.

\subsection{delete}

\subsubsection{process of delete}
how to handle single delete, single delete with selector and segmented delete

\subsubsection{special selector in delete}
selector not to select one data but all the data conforming to the constraints

\subsubsection{delete command lifetime timeout}

\subsubsection{what to do if fail to delete some of the data packets}

\subsubsection{how to build application using delete command}
ndndeletefile

\subsection{watch prefix}

\subsubsection{process of watch prefix}

\subsubsection{what to do if data time-out}

\subsubsection{why use watch prefix command}

\section{implementation of repo -- repo-ng}
\subsection{storage-design}
using database sqlite for back-end storage: better cross-platform, light-weight easy to implement

using skiplist as index. for better performance rebuild index whenever repo restarts

\subsection{trust model and access control}
using valdiator-conf

access control list designed referred to validator-conf design

\subsection{congestion control design}

\section{Evaluation}

\subsection{local access}
throughput compared to ccnr

\subsection{network access}
throughput of one-hop network

maybe simulation of large-scale network to test failure cases.

\section{conclusion and future work}


% that's all folks
\end{document}


